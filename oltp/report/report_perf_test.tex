\documentclass[report.tex]{subfiles}

\begin{document}

\section{Performance Tests}

In this section \textit{PostgreSQL}\cite{postgres} and \textit{CockroachDB}\cite{cockroach} will be tested under OLTP workload. To make experiment's results comparable to similar benchmarks, \textit{TPC-C}\cite{tpcc} will be used as benchmark standard.

\subsection{Benchmark Frameworks}
\subfile{tables/weights.tex}
Both benchmark frameworks which are used in this project, generate TPC-C workload regarding to rules mentioned in related document\cite{tpcc}. As Table \ref{table:weights} represents, queries' weight are set to default weights of TPC-C.

\subsubsection{OLTP-Bench}
OLTP workload for PostgreSQL is generated by \textit{OLTP-Bench}\cite{oltpbench}.

\subsubsection{CockroachDB's Built-in Tools}
OLTP workload for CockroachDB is generated using database's built-in tools\cite{workload}.

\subsection{Metrics}
\subsubsection{System Load}
In \textit{*NIX} systems, \textit{system load} represents the number of tasks in the queue of CPU. Since in tremendous workloads, CPU utilization is likely to be close to 100\%, system load is more accurate metric. System load also have been measured by \textit{Dstat}\cite{dstat}.

\subfile{charts/load.tex}
Figure \ref{fig:oltp_load} shows average system load while queries were running.

\end{document}
